%---------------------------------------------------------------------------------------------
\chapter{Introduction\footnote{Author: Tra Ngoc Nguyen.}}

\textit{This chapter aims to provide the objective and structure of this thesis.
	The following sections summarize our research goals and provide an overview of the organization and structure of this work.}
\section{Motivation}

This project is required by "Vietnam-German University" (VGU) in Project Module as a mandatory course in "Computer Science" (CS) major Bachelor Program. 
\section{Objective and Requirements}
The project will create a simulator to show how the power grid react to different simulated situations through various control mechanisms. The purpose is to replicate as closely as possible to how existing power grids react to similar conditions.
This simulator will receive initial information like: number of consumers, their maximum and minimum usage in Walt, number of generators, their maximum generated power, etc...
These information will be used to initalize the simulator and the output will be a graph to show how much electricity is produced and how much is used in a particular time in a day.
The Objective of this project is to design a simulator do the work above.
In the course of this project, we must meet the following requirements:
\newpage
\begin{description}
    \subsection{Generator Model Requirements}
        \item MUST be registered at the Control Model
        \item MUST provide an interface to measure the momentary power
        \item MUST provide the minimum and maximum supply
        \item MUST provide the maximum supply-change per iteration
        \item MUST provide an interface for the Control Model to request a supply change
        \item MAY change supply due to internal or external conditions (e.g. time, weather, price ..)
        \item MAY turn automatically off after a nr. of iterations (e.g. battery, pump power plant)
    \subsection{Consumer Model Requirements}
        \item MUST be registered at the Control Model
        \item MUST have an On and Off state, each with defined power
        \item MAY be registered as cluster with other consumers (e.g. households)
        \item MUST provide an interface to measure the power
        \item MUST provide an interface for remote changes
        \item MAY change state due to internal or external conditions  (e.g. time, weather, price ..)
    \subsection{Control Model Requirements}
        \item MUST be able register an arbitary number of generators and consumers
        \item MUST be able to un-register each component
        \item MUST compute the total demand every iteration
        \item MUST compute the total cost every iteration
        \item MUST compute the mains frequency every iteration (50Hz minus the difference of demand-supply, whereas 10% equals 1Hz)
        \item MUST unregister 10% Generators, if mains frequency > 51Hz (overload)
        \item MUST unregister 15% of the consumers if mains frequency < 49Hz (blackout)
        \item MUST unregister components if mains frequency > 51Hz < 49 Hz for 3 Iterations (defect)
        \item MAY request state changes in consumers for demand side management
        \item MAY request supply and state changes in generators for demand side management
        \item MAY request to start or shutdown generators for demand side management
        \item MAY change the electricity price
    \subsection{Graphical User Interface Requirements}
        \item MUST be capable to control current the iteration
        \item MUST be capable to show current demand, supply and frequency
        \item MUST be capable to show the number of consumer and generators
        \item MUST be capable to show current weather and electricity price
        \item MUST be capable to show name and power of individual consumer or generator

\end{description}
\section{Outline}
This report consist of 2 major part - First, a detail analysis of our software. This include software architecture, use-case analysis, class diagram.
The seccond portion will be Code explanation base on the analysis in the previous chapter.
As said. The rest of the report will be organized as follow:
\begin{description}
        \item [Chapter 2]: Software analysis
        \item [Chapter 3]: Code explanation
        \item [Chapter 4]: Conclusion
\end{description}
