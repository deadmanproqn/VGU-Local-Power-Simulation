
%---------------------------------------------------------------------------------------------
\chapter*{Abstract\footnote{Author: Tra Ngoc Nguyen.}}

\noindent

    Due to the massive expansion of renewable energy, many European countries launched a conversion to the next generation power grid. The large-scale use of distributed power generators such as wind and photovoltaic energy in low-voltage networks places new challenging demands on electricity grids: While low-voltage networks in traditional power grids are conventionally unmonitored, the decentralized nature of volatile and renewable energies results in a strong need to control and monitor actors in order to react timely to the variable energy demands: \newline
    
    The electricity grid is an unstable balance in which power generation and power consumption must balance each other at every time. When the mains frequency decreases, consumers slow down energy production more than electricity generators can generate. As the mains frequency increases, there is more power generated than needed. Thus, the mains frequency is the reference value of the grid stability. For longer-term frequency decreases, additional energy needs to be fed into the grid and, conversely, frequency increases require power plants to rapidly reduce their capacity. A loss of stability occurs when the grid is no longer able to return to a stable operating point following the occurrence of a disturbance that results in a large imbalance between power generation and load. This can then cause persistent oscillations of the grid frequency with automatic shut-downs of generating units and hence blackouts or in the worst case it can damage critical infrastructure. In order to keep the grid frequency stable at 50 Hz, it requires an intelligent supply-demand mechanism and, in the case of over-frequency or under-frequency, of a functioning control energy system, which is summarized as 'Demand Side Management' (DSM). \newline
    
    Simulations can be used to analyse and evaluate various control mechanisms to improve DSM. They allow to generate unorthodox consumer profiles or abnormal scenarios, that are not easy to reproduce in real world environments. The objective of this project is to develop a mains frequency simulator for a small metropolitan power grids. The resulting application should allow the user to simulate and visualize the stability of a power grid with various parameters and show the simulation result as a graph.

